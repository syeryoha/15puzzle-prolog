\documentclass[dvips,12pt,a4paper]{article}
\usepackage[brazilian]{babel}
\usepackage[latin1]{inputenc}
\usepackage[T1]{fontenc}
\usepackage[dvips]{graphicx}
\usepackage{pslatex}
\usepackage{graphics}
\usepackage{indentfirst}

\title{Trabalho Pr�tico 2\\Jogo Qubic\\Intelig�ncia Artificial}
\author{Leandro Soriano Marcolino\\soriano@dcc.ufmg.br \and Rog�rio Vinhal Nunes\\rogervn@dcc.ufmg.br}

\begin{document}
\maketitle
\newpage

\section{Representa��o do tabuleiro}

A representa��o do tabuleiro geral consiste em uma lista de 4 tabuleiros. Cada um desses tabuleiros consiste em uma lista de 4 linhas, onde cada uma dessas linhas possui 4 n�meros, correspondentes �s 4 colunas. Esses 4 n�meros podem ser -1, 0 ou 1, onde -1 corresponde a uma posi��o ocupada pelo inimigo, 0 corresponde a uma posi��o vazia e 1 corresponde a uma posi��o ocupada pelo computador.

Basicamente esta � a configura��o:

Tabuleiro Geral = [Tabuleiro1, Tabuleiro2, Tabuleiro3, Tabuleiro4]

Tabuleiro = [Linha1. Linha2, Linha3, Linha4]

Linha = [Coluna1, Coluna2, Coluna3, Coluna4]

Coluna = -1, 0 ou 1.

Um exemplo de tabuleiro preenchido com uma pe�a do inimigo na posi��o (0,0,0) e uma pe�a do computador na posi��o (3,3,3) est� descrito a seguir:

[[[-1,0,0,0], [0,0,0,0], [0,0,0,0], [0,0,0,0]], [[0,0,0,0], [0,0,0,0], [0,0,0,0], [0,0,0,0]], [[0,0,0,0], [0,0,0,0], [0,0,0,0], [0,0,0,0]], [[0,0,0,0], [0,0,0,0], [0,0,0,0], [0,0,0,1]]]

\section{Heur�stica}

A Heur�stica consiste de avaliar a possibilidade de uma jogada ter mais condi��es de vit�ria que as outras. Para isso, contamos o n�mero de possibilidades de conex�es que podem resultar em vit�ria para cada uma das posi��es vazias e multiplicamos possibilidades diversas de vit�rias, a fim de preferir uma jogada com mais chances de vencer.

Posi��es mais centrais no tabuleiro tamb�m possuem uma bonifica��o, j� que elas s�o posi��es mais estrat�gicas quando consideradas situa��es iguais com as demais.

\end{document}
